\documentclass[a4paper,12pt]{article}

% Paquetes
\usepackage[utf8]{inputenc}  % Codificación de caracteres
\usepackage{graphicx}        % Inclusión de gráficos
\usepackage{amsmath, amssymb} % Símbolos matemáticos
\usepackage{natbib}          % Bibliografía
\usepackage{hyperref}        % Hipervínculos
\usepackage{geometry}        % Margins
\geometry{margin=1in}       % Ajuste de márgenes

% Información del documento
\title{Indice actuarial para el calculo del clima en Colombia}
\author{Persona 1, Persona 2, Persona 3}
\date{\today}

\begin{document}

\maketitle

\begin{abstract}
Aquí va el resumen del artículo, con una breve descripción de los objetivos, métodos y hallazgos principales.
\end{abstract}

\section{Introduction}
Breve introducción al tema del. Contexto general y motivación del estudio.

\section{Data and Methods}
Descripción de los datos utilizados (e.g., EMWF) y la metodología para analizar la climatologia.

\section{Results and Discussion}
Presentación de los resultados clave, incluyendo gráficos y análisis. Discusión sobre el impacto de los resultados en el contexto de la climatología.

\section{Conclusions}
Resumen de los principales hallazgos y posibles trabajos futuros.

\section*{Acknowledgments}
Agradecimientos a instituciones y personas que contribuyeron al trabajo.

\bibliographystyle{apalike} % Estilo de bibliografía
\bibliography{references} % Archivo de referencias (crear 'references.bib')

\end{document}